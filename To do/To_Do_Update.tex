\documentclass[11pt, french]{article}
\usepackage{amsmath,amssymb,latexsym}

%\usepackage{palatino}
%\usepackage{arev}
%\usepackage{fourier}
%\usepackage{antpolt}

\usepackage{graphicx}
\usepackage{tikz-qtree}
\usepackage{amscd}
\usepackage{bbm}
\usepackage{caption}
\usepackage[utf8]{inputenc}
\usepackage[T1]{fontenc}
\usepackage{babel}
\usepackage{rotating,amssymb,amsmath,amsfonts,delarray}
\usepackage{geometry}
\usepackage{placeins}
\usepackage{caption}
\usepackage{float}
\usepackage{amsthm}
\newtheorem*{remark}{Remark}
\geometry{hmargin=80pt, vmargin=70pt}
\usepackage{amsfonts}
\usepackage{xcolor}
\usepackage{hyperref}
\usepackage{enumerate}
\usepackage{enumitem}
\usepackage{tikz}
\usepackage{cancel}
\usepackage{scalerel,stackengine}

\usepackage{subfig}
\usepackage{forest}

\colorlet{mygreen}{green!75!black}
\colorlet{col1in}{red!30}
\colorlet{col1out}{red!40}
\colorlet{col2in}{mygreen!40}
\colorlet{col2out}{mygreen!50}
\colorlet{col3in}{blue!30}
\colorlet{col3out}{blue!40}
\colorlet{col4in}{mygreen!20}
\colorlet{col4out}{mygreen!30}
\colorlet{col5in}{blue!10}
\colorlet{col5out}{blue!20}
\colorlet{col6in}{blue!20}
\colorlet{col6out}{blue!30}
\colorlet{col7out}{orange}
\colorlet{col7in}{orange!50}
\colorlet{col8out}{orange!40}
\colorlet{col8in}{orange!20}
\colorlet{linecol}{blue!60}

\hypersetup{
  colorlinks=true,
  citecolor=blue,
  linkcolor=blue,
  filecolor=magenta,
  urlcolor=cyan,
}

\newtheorem{rem}{Remark}
\newtheorem{deff}{Definition}
\newtheorem{lem}{Lemma}
\newtheorem{ex}{Example}
\newtheorem{theorem}{Theorem}
\newtheorem{algo}{Algorithm}

\usepackage{soul}

\pagenumbering{gobble}

\setlength{\parindent}{0cm}

\setlength\parindent{0pt}

\begin{document}

\begin{figure}[H]
  \centering
  \begin{tabular}{||c|c|c||}
  \hline
  {} & \textbf{Fonction} & \textbf{Description} \\
  \hline
  \texttt{AMH} & X &\\
  \hline
  \texttt{Clayton} & X &  \\
  \hline
  \texttt{Frank} & X &  \\
  \hline
  \texttt{GAMMA} & X & \\
  \hline
  \texttt{GeneticCodes} & X & X \\
  \hline
  \texttt{GEO} & X & \\
  \hline
  \texttt{Gumbel} & & \\
  \hline
  \texttt{InvLap} & X & \\
  \hline
  \texttt{Lap} & X & \\
  \hline
  \texttt{LOG} &  & \\
  \hline
  \texttt{Node} & X & \\
  \hline
  \texttt{pCompCop} & X & \\
  \hline
  \texttt{pCop} & X & \\
  \hline
  \texttt{rCompCop} & X & \\
  \hline
  \texttt{rCompCop2} & X & \\
  \hline
  \texttt{rCop} & X & \\
  \hline
  \texttt{rStruc} & & \\
  \hline
  \end{tabular}
\end{figure}





\end{document}
